\documentclass{article}

\usepackage[latin1]{inputenc}
\usepackage{tikz}
\usetikzlibrary{shapes,arrows, shadows}

%%%<
\usepackage{verbatim}
\usepackage[active,tightpage]{preview}
\PreviewEnvironment{tikzpicture}
\setlength\PreviewBorder{5pt}%
%%%>
\tikzset{%
  cascaded/.style = {%
    general shadow = {%
      shadow scale = 1,
      shadow xshift = -1ex,
      shadow yshift = 1ex,
      draw,
      thick,
      fill = blue!20},
    general shadow = {%
      shadow scale = 1,
      shadow xshift = -.5ex,
      shadow yshift = .5ex,
      draw,
      thick,
      fill = blue!20},
    fill = blue!20, 
    text width=4.5em,
	text badly centered,
    draw,
    thick,
    minimum width = 1.5cm,
    minimum height = 2cm}}
    
\tikzset{    
  multidocument/.style={
    shape=tape,
    text width=4.5em,
    node distance=3cm,
    text badly centered,
    draw,
    fill=blue!20,
    tape bend top=none,
    double copy shadow}
}


\begin{comment}
:Title: Simple flow chart
:Tags: Diagrams

With PGF/TikZ you can draw flow charts with relative ease. This flow chart from [1]_
outlines an algorithm for identifying the parameters of an autonomous underwater vehicle model. 

Note that relative node
placement has been used to avoid placing nodes explicitly. This feature was
introduced in PGF/TikZ >= 1.09.

.. [1] Bossley, K.; Brown, M. & Harris, C. Neurofuzzy identification of an autonomous underwater vehicle `International Journal of Systems Science`, 1999, 30, 901-913 


\end{comment}


\begin{document}
\pagestyle{empty}


% Define block styles
\tikzstyle{decisiono} = [diamond, draw, fill=blue!20, 
    text width=4.5em, text badly centered, node distance=3cm, inner sep=0pt]
\tikzstyle{block} = [rectangle, draw, fill=blue!20, 
    text width=5em, text centered, rounded corners, minimum height=4em]
\tikzstyle{line} = [draw, -latex']
\tikzstyle{cloud} = [draw, ellipse,fill=red!20, node distance=3cm,
    minimum height=2em]
    
\begin{tikzpicture}[node distance = 2cm, auto]
    % Place nodes	
    \node [block] (remove) {Clean Wikipedia};
    \node [multidocument, left of=remove](dump) {Wikipedia Dump};

    \node [block, below of=remove] (parse) {Parse Wikipedia};
    \node [draw, right of=parse, node distance=3cm] at (0,-1) (token) {Identify Tokens};
    \node [draw, right of=parse, node distance=3cm] at (-.5,-2) (pos) {PoS tags};
	\node [draw, right of=parse, node distance=3cm] at (0,-3) (sparse) {Syntactic Parse};
    \node [block, below of=parse] (output) {{Store Wikipedia Parse}};
	\node [cascaded, below of=output, node distance=2.5cm](pdump) {YAWPD};

    % Draw edges
    \path [line] (remove) -- (parse);
    \path [line,dashed] (parse) --  (token.west);
    \path [line,dashed] (parse) --  (pos);
    \path [line,dashed] (parse) --  (sparse.west);
    \path [line] (parse) -- (output);
    \path [line,shorten >=4pt] (output) -- (pdump.north);
   % \path [line] (evaluate) -- (decide);
 %   \path [line] (decide) -| node [near start] {yes} (update);
%    \path [line] (update) |- (identify);
    %\path [line] (decide) -- node {no}(stop);
    \path [line,shorten <=4pt] (dump) -- (remove);

\end{tikzpicture}


\end{document}